%%%%%%%%%%%%%%%%%%%%%%%%%%%%%%%%%%%%%%%%%
% Beamer Presentation
% LaTeX Template
% Version 1.0 (10/11/12)
%
% This template has been downloaded from:
% http://www.LaTeXTemplates.com
%
% License:
% CC BY-NC-SA 3.0 (http://creativecommons.org/licenses/by-nc-sa/3.0/)
%
%%%%%%%%%%%%%%%%%%%%%%%%%%%%%%%%%%%%%%%%%

%----------------------------------------------------------------------------------------
%	PACKAGES AND THEMES
%----------------------------------------------------------------------------------------

\documentclass[handout]{beamer}

\mode<presentation> {

% The Beamer class comes with a number of default slide themes
% which change the colors and layouts of slides. Below this is a list
% of all the themes, uncomment each in turn to see what they look like.

%\usetheme{default}
%\usetheme{AnnArbor}
\usetheme{Antibes}
%\usetheme{Bergen}
%\usetheme{Berkeley}
%\usetheme{Berlin}
%\usetheme{Boadilla}
%\usetheme{CambridgeUS}
%\usetheme{Copenhagen}
%\usetheme{Darmstadt}
%\usetheme{Dresden}
%\usetheme{Frankfurt}
%\usetheme{Goettingen}
%\usetheme{Hannover}
%\usetheme{Ilmenau}
%\usetheme{JuanLesPins}
%\usetheme{Luebeck}
%\usetheme{Madrid}
%\usetheme{Malmoe}
%\usetheme{Marburg}
%\usetheme{Montpellier}
%\usetheme{PaloAlto}
%\usetheme{Pittsburgh}
%\usetheme{Rochester}
%\usetheme{Singapore}
%\usetheme{Szeged}
%\usetheme{Warsaw}

% As well as themes, the Beamer class has a number of color themes
% for any slide theme. Uncomment each of these in turn to see how it
% changes the colors of your current slide theme.

%\usecolortheme{albatross}
%\usecolortheme{beaver}
%\usecolortheme{beetle}
%\usecolortheme{crane}
%\usecolortheme{dolphin}
%\usecolortheme{dove}
%\usecolortheme{fly}
%\usecolortheme{lily}
%\usecolortheme{orchid}
%\usecolortheme{rose}
%\usecolortheme{seagull}
%\usecolortheme{seahorse}
%\usecolortheme{whale}
%\usecolortheme{wolverine}

%\setbeamertemplate{footline} % To remove the footer line in all slides uncomment this line
%\setbeamertemplate{footline}[page number] % To replace the footer line in all slides with a simple slide count uncomment this line

%\setbeamertemplate{navigation symbols}{} % To remove the navigation symbols from the bottom of all slides uncomment this line
}

\usepackage{graphicx} % Allows including images
\usepackage{booktabs} % Allows the use of \toprule, \midrule and \bottomrule in tables
\usepackage{hyperref}
%----------------------------------------------------------------------------------------
%	TITLE PAGE
%----------------------------------------------------------------------------------------

\title[Similarities between three texts]{The similarities between \\ Dr. Martin Luther King Jr.'s \\Letter from Birmingham jail, \\ Malala Yousafzai's Adress to the UN, and\\ George Saunders' commencement speech 2013} % The short title appears at the bottom of every slide, the full title is only on the title page

\author{Theo Tubbs} % Your name
\institute[NCHS] % Your institution as it will appear on the bottom of every slide, may be shorthand to save space
{
Norwich City High School \\ % Your institution for the title page
\medskip
\textit{\href{mailto:adriankoshcha@vivaldi.net}{adriankoshcha@vivaldi.net}} % Your email address
}
\date{\today} % Date, can be changed to a custom date

\begin{document}

\begin{frame}
\titlepage % Print the title page as the first slide
\end{frame}

\begin{frame}
\frametitle{Table of Contents} % Table of contents slide, comment this block out to remove it
\tableofcontents % Throughout your presentation, if you choose to use \section{} and \subsection{} commands, these will automatically be printed on this slide as an overview of your presentation
\end{frame}

%----------------------------------------------------------------------------------------
%	PRESENTATION SLIDES
%----------------------------------------------------------------------------------------

%------------------------------------------------
\section{Introduction} % Sections can be created in order to organize your presentation into discrete blocks, all sections and subsections are automatically printed in the table of contents as an overview of the talk
%------------------------------------------------



\begin{frame}
\frametitle{Overview}
\begin{itemize}
\item Letter from Birmingham Jail
  \begin{enumerate}
    \item[Author:] Martin Luther King Jr.
    \item[Published:] April 16, 1963
    \item[Subject:] Civil Rights
  \end{enumerate}
\item Adress to the UN
  \begin{enumerate}
    \item[Author:] Malala Yousafzai
    \item[Published:] July 12, 2013
    \item[Subject:] Educational Rights and Equal oppurnity
  \end{enumerate}
\item commencement Speech
  \begin{enumerate}
	\item[Author:] George Saunders
    \item[Published:] May 11, 2013
    \item[Subject:] Kindness
  \end{enumerate}
\end{itemize}
\end{frame}
\note{MLKJ wrote letter while he was in jail. \\~\\ Malala was shot in the head by the Taliban but survived. \\~\\ George Saunders is a well known Author.}
%------------------------------------------------

%------------------------------------------------
\subsection{Core} % A subsection can be created just before a set of slides with a common theme to further break down your presentation into chunks
\begin{frame}
\frametitle{Pathos}
\begin{itemize}
\item Examples
	\begin{enumerate}
		\item[LFBJ:] "Injustice anywhere is a threat to justice everywhere. We are caught in an inescapable network of mutuality, tied in a single garment of destiny."
		\item[ATTUN:] "There are hundreds of human rights activists and social workers who are not only speaking for their rights, but who are struggling to achieve their goal of peace, education, and equality."
		\item[CS:] "I could see the this hurt her. I still remember the way she'd look after such an insult: eyes cast down, a little gut-kicked,...she was trying as much as possible...to disappear."
	\end{enumerate}
\end{itemize}
\end{frame}
\note{MLKJ is using parallel construction to create a powerful emotional pull on the audience. \\~\\ Malala uses the suffering of other activists to give the audience a sort of stage for understanding the struggle her(and other activists) go through. \\~\\ George Saunders takes advantage of the fact that most people have experienced or have had an experience/encounter with bullying.}
%------------------------------------------------

%------------------------------------------------
\begin{frame}
\frametitle{Exemplification}
	\begin{itemize}
		\item Examples
			\begin{enumerate}
				\item[LFBJ:] "It's unjust treatment of Negros in the courts if a notorious reality."
				\item[ATTUN:] "The extremists are afraid of books and pens. The power of education frightens them."
				\item[CS:] "You do well in high-school, in hopes of getting into a good college, so you can do well in college, in hopes of getting a good job..."
			\end{enumerate}
	\end{itemize}
\end{frame}
\note{ MLKJ utilizes the shock value of unjust treatment to harden his point of civil inequality. \\~\\ Malala makes use of the fact that the Taliban is afraid of education to humanize them and make them less intimidating. \\~\\ George Saunders points out the cyclic nature of life to provide an empathetical standpoint for the audience.}
%------------------------------------------------

%------------------------------------------------
\begin{frame}
\frametitle{Counterargument}
       \begin{itemize}
               \item Examples
                       \begin{enumerate}
                               \item[LFBJ:] "I think I should give the reason for being here in Birmingham...I have the honor of serving as president of the Souther Christian Conference..."
                               \item[ATTUN:] "The terrorists thought they would change my aims and my ambitions. But nothing changed in my life except this: weakness, fear and hopelessness died. Strength, power and courage was born."
                               \item[CS:] "we don't really believe these things - intellectually we know better..."
		       \end{enumerate}
       \end{itemize}
\end{frame}
\note{MLKJ answers why he was in birmingham, strengthening his argument by giving it purpose.\\~\\ Malala shows her perseverance in this quote.\\~\\ George Saunders counters his own previous argument to remind the audience of what makes people 'human' \\~\\}

%------------------------------------------------

%------------------------------------------------
%\begin{frame}
%\frametitle{}
%	\begin{itemize}
%		\item Examples
%			\begin{enumerate}
%				\item[LFBJ:] ""
%				\item[ATTUN:] ""
%				\item[CS:] ""
%			\end{enumerate}
%	\end{itemize}
%\end{frame}
%\note{}
%------------------------------------------------


\end{document} 
